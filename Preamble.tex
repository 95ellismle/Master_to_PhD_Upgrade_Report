% I may change the way this is done in a future version,
%  but given that some people needed it, if you need a different degree title
%  (e.g. Master of Science, Master in Science, Master of Arts, etc)
%  uncomment the following 3 lines and set as appropriate (this *has* to be before \maketitle)
 \makeatletter
\renewcommand {\@degree@string} {Master of Philosophy to Doctor of Philosophy}
 \makeatother

\title{Developing FOB-CTMQC for use in simulating charge transport in organic semiconductors}
\author{Matt Ellis}
\department{Department of Physics and Astronomy}

\maketitle
\makedeclaration

\begin{abstract} % 300 word limit
Charge transfer in organic molecular systems are difficult to simulate due to (fast?) non-adiabatic transitions between Born-Oppenheimer energy surfaces. A range of techniques have been designed to overcome this, such as the Surface Hopping and Ehrenfest methods. However, they tend to suffer from unphysical overcoherence issues. In this report, I present an implementation of a fragment-orbital based coupled-trajectory mixed quantum-classical algorithm (FOB-CTMQC), which is designed for simulating charge transport in systems of tens to hundreds of organic molecules. This method uses an in-house analytical overlap method (AOM) within the framework of coupled-trajectory mixed quantum-classical (CTMQC) molecular dynamics. CTMQC enables the correct calculation of decoherence due to 2 new terms containing a quantity named the Quantum Momentum. While the AOM method allows the simulation of large systems, due to the method's analytic formulation of the off-diagonal elements of the Hamiltonian in terms of the overlap between singly occupied molecular orbitals. E.g. $H_{kl} = CS_{kl} $, where $H_{kl}$ are the off-diagonal elements of the Hamiltonian, C is a constant of proportion and $S_{kl}$ are the off-diagonal overlap elements. For many organic semiconductors we can assume pi-conjugation. Meaning only 1 optimized Slater p-orbital is needed for the calculation of the overlap. 
\end{abstract}

\setcounter{tocdepth}{2}
% Setting this higher means you get contents entries for
%  more minor section headers.

\tableofcontents
%\listoffigures
%\listoftables
