% I may change the way this is done in a future version,
%  but given that some people needed it, if you need a different degree title
%  (e.g. Master of Science, Master in Science, Master of Arts, etc)
%  uncomment the following 3 lines and set as appropriate (this *has* to be before \maketitle)
 \makeatletter
\renewcommand {\@degree@string} {Master of Philosophy to Doctor of Philosophy}
 \makeatother

\title{Implementation of Coupled-Trajectory Mixed Quantum Classical molecular dynamics for simulation of charge transport in organic and biological materials.    }
\author{Matt Ellis}
\department{Department of Physics and Astronomy}

\maketitle
\begin{abstract} % 300 word limit
Non-adiabatic processes play a vital role in many interesting processes such as photosynthesis, respiration and electronic charge transfers. In these systems the Born-Oppenheimer approximation cannot be applied and one must use computational simulations to model the physics. Many techniques have been proposed for this, all of which strike a compromise between accuracy and computation efficiency. To deal with realistic systems that have hundreds to thousands of molecules one must use mixed quantum-classical dynamics where the slow degrees of freedom (the nuclei) are treated classically and the fast ones treated with quantum mechanics. A range of techniques such as the Surface Hopping and Ehrenfest methods have been designed with this in mind. However, they tend to suffer from unphysical over-coherence issues. In this report, I present an implementation of a fragment-orbital based coupled-trajectory mixed quantum-classical algorithm (FOB-CTMQC), which is designed for simulating charge transport in systems of hundreds to thousands organic molecules. This method uses an in-house analytical overlap method (AOM) within the framework of coupled-trajectory mixed quantum-classical (CTMQC) molecular dynamics. CTMQC incorporates decoherence due to two new terms containing a quantity named the Quantum Momentum. I present several numerical tests of my implementation of the method. These include a demonstration of Rabi oscillation, energy conservation and a test of an essential equation used in the derivation of the CTMQC equations. I also discuss a problem in the norm conservation and an implementation of a smoothing function designed to overcome this.
\end{abstract}

\setcounter{tocdepth}{2}
% Setting this higher means you get contents entries for
%  more minor section headers.

\tableofcontents
%\listoffigures
%\listoftables
