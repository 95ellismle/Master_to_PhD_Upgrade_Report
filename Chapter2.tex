\chapter{FOB Formalism}
\label{chap:FOB}
\subsection{FOB Formalism \label{sec:FOB-formalism}}
The effect of the nuclei on the electrons is normally handled via the Hamiltonian. This is dependent on nuclear positions and is in turn used in the Schr\"odinger equation to propagate the electron dynamics. Often the construction of the Hamiltonian is carried out using density function theory (DFT). However, for large, dynamic systems this becomes too computationally expensive and a different technique would have to be used. In this work I will rely on an Analytical Overlap Method (AOM) \cite{gajdos_ultrafast_2014} to calculate the off-diagonal elements of the Hamiltonian and the diagonal elements will be calculated via a forcefield.
\subsubsection{AOM}
AOM assumes that the off-diagonal elements of the Hamiltonian are proportional to the off-diagonal elements of the overlap matrix between 2 singly occupied molecular orbitals (SOMO). For $\pi$-conjugated systems,
