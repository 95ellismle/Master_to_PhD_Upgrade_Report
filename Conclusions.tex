\chapter{General Conclusions}
\label{Conclusions}
There are many real world applications of organic semiconductors and accurate models of charge transport are important to facilitate new materials discovery and characterisation. However, due to mobilities falling within an intermediate region where neither band theories nor hopping theories are applicable non-adiabatic atomistic simulations must be used. Among the litany of techniques proposed there are no single silver bullets. As always, the user must make the compromise between accurate dynamics and computational cost. Two of the most popular mixed-quantum classical techniques are trajectory surface hopping (TSH) and Ehrenfest. However, these both suffer from well known problems such as over-coherent nuclear-electronic dynamics  no branching of the nuclear wavefunction in Ehrenfest and lack of a first-principles grounding in TSH.
\\\\
To overcome these challenges a new technique, coupled-trajectory mixed-quantum classical molecular dynamics (CTMQC), has been proposed \cite{agostini_semiclassical_2015} to more rigorously account for decoherence, branching of the nuclear wavefunction and to provide a technique based in first principles physics. This technique, derived from the exact factorisation of the molecular wavefunction \cite{abedi_exact_2010}, appears as a `corrected' Ehrenfest scheme where the correction comes from 2 new terms -an adiabatic time-integrated force and a quantum momentum.
\\\\
In this report I have outlined an implementation of CTMQC paired with a fragment-orbital based (FOB) technique to produce an efficient FOB-CTMQC propagator capable of simulating hundreds of organic molecules. The FOB method is based on the assumption that the electronic couplings (off-diagonal Hamiltonian elements) are proportional to the overlap between singly occupied molecular orbitals (SOMOs). This approximation has been validated in many organic semiconductors and provides a significant speed-up when compared to using density functional theory.
\\\\
The implementation process is still under progress. However, initial results are promising. Some key tests have been discussed including Rabi oscillation, energy conservation, norm conservation and the fulfilment of a fundamental equation \eqref{eq:S26}. The tests have been mostly positive. However, due a denominator in the equation for calculating the quantum momentum approaching zero -causing the quantum momentum to spike. The norm was not well conserved and a smoothing tanh$^2$ function was used to fix this.
\\\\
To build on this work I would now like to apply FOB-CTMQC to more realistic systems and to eventually compare with experimental results. However, a number of tasks must be completed before this is possible. These include implementing a sensible algorithm for calculating the nuclear width parameter, $\sigma_{\nu}^{I}$, used in the calculation of the quantum momentum. This determines the width of the gaussians that combine to give the quantum momentum. I am currently using a frozen width of $\sqrt{2}$ bohr. The norm conservation and the spikes in the quantum momentum should be monitored also, if these get worse an improved smoothing algorithm will have to be implemented. Over the next year I would like to also test whether detailed balance is reached in CTMQC as well as comparing CTMQC with our in-house FOB-SH algorithm as well as with more accurate methods. CTMQC should be more accurate than the surface hopping (FOB-SH) algorithm and handle decoherence in a better way. For smaller systems CTMQC can be benchmarked against more accurate results such as the multi-configuration time-dependent Hartree method discussed here \cite{cattarius_all_2001}. Finally many optimisations will have to be implemented before moving to realistic systems consisting of hundreds of molecules or more. One of the most important of these is the parallelisation of the code in order to efficiently use multiple processors.